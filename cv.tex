\documentclass[helvetica,notitle,nologo]{europecv}
\usepackage[a4paper, total={6in, 8in}]{geometry}
\usepackage{graphicx}
\usepackage{tikz}
\usepackage[toc, page]{appendix}

\newcommand{\grade}[1]{%
\begin{tikzpicture}
\clip (1em-.3em,-.3em) rectangle (5em +.5em ,.3em);
\begin{scope}
\clip (1em-.3em,-.3em) rectangle (#1em +.5em ,.3em);
\foreach \x in {1,2,...,5}{
 \path[fill=cyan] (\x em,0) circle (.25em);
}
\end{scope}
\foreach \x in {1,2,...,5}{
 \draw (\x em,0) circle (.25em);
}
\end{tikzpicture}%
}
\ecvfootername{Alexander Premer}
\begin{document}
\begin{europecv}

\ecvsection{Personlig Informasjon}
\ecvitem{Navn}{\textbf{Alexander Premer}}
\ecvitem{Alder}{27 år}
\ecvitem{Tlf.}{+47 942 38 318}
\ecvitem{Epost}{Alex.premer@gmail.com}
\ecvitem{}{}
\ecvitem{Om}{
   \textit{Alexander er en nysgjerrig og lærenem utvikler, som engasjerer seg
   for funksjonnelle språk som f.eks Scala. Alexander legger vekt på
   'clean-code'-prinsipper og lesbar kode. Alexander har sin kjernekompetanse
   innenfor systemutvikling og har den siste tiden hatt fokus på mikrotjenester
   og smidig utvikling. Han bidrar til sine prosjekter med sin sympatiske og
   løsningsorienterte holdning. På fritiden liker Alexander godt å spille
   sjakk, frisbee-golf, squash og brettspill.}
}

\ecvsection{Arbeidserfaring}
\ecvitem{Feb. 2018 - dd\\ \textbf{Konsulent - Knowit}}{
   \textbf{Entur - Modernisering salgssystem (SPRANG):} \textit{Alexander
   er en viktig ressurs i et tverrfaglig autonomt team som jobber med
   modernisering av billetteringssystem for tog og øvrig kollektiv hos Entur.
   Alexander har fokus på god kodekvalitet og utvikling av Enturs produkter.}
}
\ecvitem{}{}
\ecvitem{Sept 2016 - Jan. 2018\\ \textbf{Konsulent - Knowit}}{
   \textbf{Oslo Kommune - Program for Felles Funksjonalitet:} \textit{Alexander
   startet i prosjektet som backend systemutvikler med hovedfokus på utvikling
   av ny funksjonalitet på min side. Siden Oktober 2017 har Alexanders ledet et
   autonomt og tverrfaglig team i en 50-50prosent stilling som teamleder og
   utvikler. Arbeidet omfatter ansvar for analyse/design av løsning,
   estimering, avklaringer med kunde og utviklere i teamet, kvalitetssikring,
   oppfølging og rapportering.}
}
\ecvitem{}{}
\ecvitem{Mars 2014 - Jan. 2015}{
   \textbf{IKT-konsulent Sykehuspartner:} \textit{
      Alexander assisterte helsearbeidere i Helse Sør-Øst med IT problemer i
      deres hektiske arbeidsdag. Alexander var på førstelinje og tok imot de
      problemene brukerne hadde og løste disse problemene på en mest mulig
      effektiv måte.}
}
\ecvitem{}{}
\ecvitem{Des. 2011 - Aug. 2012}{
   \textbf{Buttikkmedarbeider på Shell/7-eleven, Tåsen} \textit{Alexander jobbet som
   regel natt og jobbet hardt for å holde en viss standard hos bedriften ved å
   fylle på varer, vaske lokalene og betjene kunder.}
}
\ecvitem{}{}
\ecvitem{Okt. 2008 - Sept. 2011}{
   \textbf{Buttikkmedarbeider på ICA, Alexander Kiellands Gate: }
   \textit{Alexander hadde ansvar for kjøkkentjenester, tippedisk, varesalg,
   varepåfylling og overholde en god standard i butikken.}
}
\ecvitem{}{}
\ecvitem{Feb. 2008 - Aug. 2008}{
   \textbf{Buttikkmedarbeider på ICA, Flateby: } \textit{Alexander hadde ansvar
   varepåfylling og overholde en god standard i butikken.}
}

\ecvsection{Utdannelse}
\ecvitem{Universitetet i Oslo\\Aug. 2014- Aug. 2016}{
   \textbf{Master Grad i Informatikk: Programmering og Nettverk: } \textit{
   Etter hvert som helse informasjons systemer i utviklingsland har blitt mer
   og mer vanlig, har akademikere identifisert at systemene ofte ikke skalerer
   og landene er forlatt med prototyper som er vanskelig å skalere og
   videreutvikle etter at donororganisasjoners støtte forsvinner. Alexanders
   mastergrad handlet om å utforske hvordan de involverte kunne legge til rette
   for en sukessfull implementasjon av helse informasjonssystemer i
   utviklingsland.}
}
\ecvitem{}{}
\ecvitem{San Jose State University\\Aug. 2012- Des. 2012}{
   \textbf{Ett semester som utvekslingsstudent}
}
\ecvitem{}{}
\ecvitem{Universistetet i Oslo\\Aug. 2011- Juli 2014}{
   \textbf{Bachelor grad i Informatikk: Programmering og Nettverk: } \textit{
      Etter et semester på Høgskolen i Oslo-Akershus fant Alexander ut at
      programmering var noe han ville fortsette med, men at han heller ville
      starte på UiO.}
}
\ecvitem{}{}
\ecvitem{Skedsmo VGS\\Aug. 2007- Juli 2010}{\textbf{Studiespesialiserende: Realfag}}

\ecvsection{Frivillig Arbeid}
\ecvitem{2015}{\textbf{Frivillig på JavaZone : } Java Konferanse}
\ecvitem{Sept. 2015 - Mai 2016}{
   \textbf{Grunnlegger \& styremedlem av IFI-Sjakk: }\textit{
      Studentorganisasjon med fokus på å danne et sjakkmiljø på fakultetet.}
}
\ecvitem{Mars 2015 - Mars 2016}{
   \textbf{Grunnlegger \& styremedlem av IFI-ProgNett: }\textit{
      Studentorganisasjon med fokus på å skape et godt læringsmiljø for
      studentene på programmering og nettverk linjen.}
}
\ecvitem{Feb. 2013 - Mai 2013}{
   \textbf{Grunnlegger \& styremedlem av IFI-ProgNett: }\textit{PR-Ansvarlig}
}
\ecvitem{2013}{
   \textbf{Frivillig på JavaZone : }\textit{Java Konferanse}
}
\ecvitem{2013}{
   \textbf{Dagen@ifi: }\textit{Oslo's største karrieredag}
}

\ecvsection{Verktøy \& Teknologi}
\ecvitem{Entur - SPRANG}{
   \textit{Scala, PostgreSQL, MongoDB, Parprogrammering, IntelliJ IDEA, Jetty,
   Gradle, ScalaTest, Scalatra, vim, Jenkins, Redis, JIRA, Smidige metoder,
   Bitbucket, Docker, Scrumban, Kubernetes, Prometheus, Akka, Grafana, CircleCI,
   ScalikeJDBC, Pacts, Google Cloud, Jackson, Memorystore.}
}
\ecvitem{}{}
\ecvitem{Oslo Kommune - PFF}{
   \textit{Java, Scala, Oracle, Testdrevet utvikling, Parprogrammering, Maven,
   IntelliJ IDEA, JIRA, Hudson/Jenkins, Mac OS, Mockito, ScalaTest, Scalatra,
   Puppet, JDBC, vim, Oracle SQL Developer, Github, Scrum, Testing,
   Smidige metoder, Mikrotjenester, Kontinuerlig deployering, Jackson, Json4s.}
}
\ecvitem{}{}
\ecvitem{Universitetet i Oslo}{
   \textit{Java, Android, \LaTeX, Assembler, C\#, JavaScript, jQuery, HTML, CSS,
   Git, Scheme, C, Jackson, Python, Linux, Hibernate, Spring, Maven, TomCat,
   Gradle.}
}

\end{europecv}
\end{document}
